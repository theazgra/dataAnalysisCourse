\documentclass{beamer}%[9pt]
\usetheme[
    sectionpage=progressbar,
    progressbar=frametitle,
    background=light
   ] {metropolis} 


\usepackage[czech]{babel}
\usepackage[utf8]{inputenc}
\usepackage[T1]{fontenc}
\usepackage{graphicx}
\usepackage{hyperref}
\usepackage{siunitx}
\usepackage{graphicx}
\usepackage{tikz}
\usepackage{listings}
\usepackage{mathtools}
\usepackage[export]{adjustbox}

\let\oldfootnotesize\footnotesize
\renewcommand*{\footnotesize}{\oldfootnotesize\tiny}

\newcommand{\pmQuant}[0]{$\text{PM}_{2.5}\:$}
\newcommand{\image}[2]{\begin{figure}[ht!] \centering \includegraphics[width=#2\linewidth]{#1} \end{figure}}
\newcommand{\croppedColorMap}[1]{\begin{figure}[ht!] \centering \adjincludegraphics[width=0.9\linewidth,trim={0 {.6\width} 0 0},clip]{#1} \end{figure}}    


\newenvironment{cols}{\begin{columns}}{\end{columns}}
\newenvironment{col}[1]{\begin{column}{{#1\linewidth}}}{\end{column}}
\newenvironment{hcol}{\begin{column}{{0.5\linewidth}}}{\end{column}}

\author{Bc. Moravec Vojtěch}
\title{Regrese koncentrace \pmQuant v Pekingu \\ MAD 3 projekt}
\date{ZS 2019/2020}
\institute{Vysoká škola báňská – Technická univerzita Ostrava}

\begin{document}
\maketitle
% \begin{frame}{Cíl}
%     \begin{itemize}
%         \item Provést explorační analýzu
%         \item Připravit datasety pro regresi
%         \item Provést a ohodnotit regresi
%     \end{itemize}
% \end{frame}
% --------------------------------------------
\begin{frame}{Informace ohledně datasetu}
    \begin{itemize}
        \item 43 824 záznamů a 12 atributů
        \item Data od 1.1.2010 až do 31.12.2014
        \item Záznamy o měření koncentrace pevných části \pmQuant ve vzduchu (\SI{}{\micro\gram}/$\text{m}^3$)
        \item 6 numerických atributů + 1 cílový numerický
        \item 1 kategoriální atribut
        \item zbylé atributy reprezentují datum a čas
        \begin{itemize}
            \item vytvořili jsme nový atribut "den v týdnu"
        \end{itemize}
    \end{itemize}
\end{frame}
% --------------------------------------------
\begin{frame}{Časové atributy}
    \begin{itemize}
        \item Jak pracovat s atributy dne, měsíce, roku, hodiny, dne v týdnu?
        \item Numerický nebo kategoriální atribut?
        \item Podíváme se na závilost cílového atributu, vzhledem k těmto atributům
        \begin{itemize}
            \item Nalezneme-li závislost (např. koncentrace roste s měsícem) - numerický
            \item Jinak kategoriální a provedeme jejich binarizaci
        \end{itemize}
    \end{itemize}
\end{frame}
% --------------------------------------------
\begin{frame}{Koncentrace vzhledem k měsíci}
    Nevidíme žádnou závislost, že koncentrace klesá nebo roste spolu s měsícem.
    \image{../target_by_month_day.pdf}{1.0}
\end{frame}
% --------------------------------------------
\begin{frame}{Koncentrace vzhledem k dnu týdne}
    Koncentrace \pmQuant je nezávislá na dnu týdne.
    \image{../target_by_week_day.pdf}{1.0}
\end{frame}
% --------------------------------------------
\begin{frame}{Odlehlé pozorování}
    Odlehlé pozorování nalezené pro:
    \begin{itemize}
        \item Rychlost větru (4 893) \footnotemark[1]
        \item Doba deště (1 739) \footnotemark[1]
        \item Doba sněžení (368) \footnotemark[1]
        \item Koncentrace \pmQuant (1 773) - všechny odstraněny
    \end{itemize}
    
    \footnotetext[1]{Odlehlé pozorování byly ponechýny v některých datasetech}
\end{frame}
% --------------------------------------------
\begin{frame}{Histogram doba sněžení, deště}
    \image{../hrain.pdf}{0.65}
    \image{../hsnow.pdf}{0.65}
    % \begin{cols}
    %     \begin{hcol}
    %     \end{hcol}
    %     \begin{hcol}
    %     \end{hcol}
    % \end{cols}
\end{frame}
% --------------------------------------------
\begin{frame}{Datasety pro regresi}
    \begin{itemize}
        \item Ve všech datasetech byla provedena normalizace hodnot do rozmezí 0,0 až 1,0
    \end{itemize}
    \begin{table}
        \centering
        \begin{tabular}{l | r | r}
            Dataset                         & Počet transakcí   & Počet attributů \\\hline\hline
            df\_binAll                      & 39984             & 89 \\
            df\_binNoDay                    & 39984             & 58 \\
            df\_noOut\_binNoDay             & 33512             & 58 \\
            df\_noOut\_binNoDay\_20attr     & 33512             & 20 \\  
            df\_noOut\_binNoDay\_10attr     & 33512             & 10 \\
        \end{tabular}
    \end{table}
\end{frame}
% --------------------------------------------
\begin{frame}{Regrese pomocí SVR}
    \croppedColorMap{../regressionResults/svr_r2.pdf}
\end{frame}
% --------------------------------------------
\begin{frame}{Regrese pomocí DecisionTreeRegressor}
    \croppedColorMap{../regressionResults/regTree_r2.pdf}
\end{frame}
% -------------------------------------------- 
\begin{frame}{Regrese pomocí MLPRegressor}
    \croppedColorMap{../regressionResults/nn_r2.pdf}
\end{frame}
% --------------------------------------------
\begin{frame}{Sourhn výsledků}
    \begin{table}
        \centering
        \begin{tabular}{l | r | r | r  | r}
            Algoritmus              & Čas učení (s) &  $R^2$ & MSE \\\hline\hline
            SVR                     & 27,209        & 0,305  & 3326,722  \\
            DecisionTreeRegressor   & 0,078         & 0,229  & 3731,770  \\
            MLPRegressor            & 18,880        & 0,336  & 3172,276
        \end{tabular}
    \end{table}
\end{frame}
% --------------------------------------------
\begin{frame}{Ansámbl metody}
    \begin{itemize}
        \item RandomForestRegressor vedl pouze k minimálnímu zlepšení
        \item Nejlepší boostovací metody seřazeny podle zlepšení (od nejlepšího):
        \begin{enumerate}
            \item HistGradientBoostingRegressor
            \item GradientBoostingRegressor
            \item AdaBoostRegressor
        \end{enumerate}
    \end{itemize}
\end{frame}
% --------------------------------------------
\begin{frame}{Boostovací metody - HistGradientBoostingRegressor}
    \croppedColorMap{../regressionResults/hgbr_r2.pdf}
\end{frame}
% --------------------------------------------
\begin{frame}{Souhrn všech regresorů}
    \croppedColorMap{../regressionResults/ultimate.pdf}
\end{frame}
% --------------------------------------------
\begin{frame}{Závěr}
    \begin{itemize}
        \item Provedli jsme explorační analýzu
        \item Vytvořili jsme datasety
        \item Otestovali jsme regresory a zhodnotili jsme je
        \item Výsledky regresorů jsou podprůměrné
        \item Koncentrace \pmQuant je těžko předpovidatelná
    \end{itemize}
\end{frame}
% --------------------------------------------
\section*{Děkuji za pozornost \\ Otázky?}
\begin{frame}
    \bibliography{../citations}
    \bibliographystyle{ieeetr}
\end{frame}
\end{document}
\end{frame}
