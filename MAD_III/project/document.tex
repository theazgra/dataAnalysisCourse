\documentclass[a4paper,12pt]{article}

\usepackage[section]{placeins}
\usepackage[czech]{babel}
\usepackage[utf8]{inputenc}
\usepackage[T1]{fontenc}
\usepackage{listings}
\usepackage{hyperref}
\usepackage{enumitem}
\usepackage{mathrsfs}
\usepackage{enumitem}
\usepackage{amsmath}
\usepackage{amssymb}
\usepackage{graphicx}
\usepackage{tikz}
\usepackage{pgfplots}
\usepackage{dirtytalk}
\usepackage{siunitx}
\usepackage{subcaption}
\usepackage{gensymb}




%\usepackage{titlesec}
%\newcommand{\sectionbreak}{\clearpage}
%\setlist{nosep}

\newcommand{\pmQuant}[0]{$\text{PM}_{2.5}$}
\newcommand{\image}[4]{\begin{figure}[ht!] \centering \includegraphics[width=#4\linewidth]{#1} \caption{#2} \label{#3} \end{figure}}
\pgfplotsset{
	compat=1.9, 
	width=7cm,
	/pgfplots/ybar legend/.style={
    /pgfplots/legend image code/.code={%
       \draw[##1,/tikz/.cd,yshift=-0.25em]
        (0cm,0cm) rectangle (3pt,0.8em);},
   },
   /pgf/number format/use comma
}
\pgfplotsset{%
    axis line origin/.style args={#1,#2}{
        x filter/.append code={ % Check for empty or filtered out numbers
            \ifx\pgfmathresult\empty\else\pgfmathparse{\pgfmathresult-#1}\fi
        },
        y filter/.append code={
            \ifx\pgfmathresult\empty\else\pgfmathparse{\pgfmathresult-#2}\fi
        },
        xticklabel=\pgfmathparse{\tick+#1}\pgfmathprintnumber{\pgfmathresult},
        yticklabel=\pgfmathparse{\tick+#2}\pgfmathprintnumber{\pgfmathresult}
    }
}
\author{Bc. Moravec Vojtěch}
\title{MAD 3 projekt \\ Regrese koncentrace pevných látek v Pekingu }
\date{ZS 2019/2020}


\begin{document}
\maketitle
\newpage

\tableofcontents
\newpage

\section{Popis problému}
Námi vybraný dataset \cite{beijing_data} se zaměřuje na znečištění vzduchu v Pekingu od 1. Ledna 2010 do 31. Prosince 2014.
Tento dataset jsme získali z UCI Machine Learning Repository \cite{Dua:2019}.
Znečištěním rozumíme koncentraci v mikrogramech na metr krychlový (\SI{}{\micro\gram}/$\text{m}^3$) pevných částic ve vzduchu.
V našem připadě se jedná o částice $\text{PM}_{2.5}$, jejíchž průměr je maximálně \SI{2,5}{\micro\metre}.
Našim cílem je tedy provést explorační analýzu datasetu a následně provést předpověď, regresi koncentrace znečištění
vzhledem k času a přiloženým meteorologickým datům.  

\section{Popis datasetu}
Dataset obsahuje celkem 43 824 záznamů a 12 atributů, nepočítáme-li číslo řádku. Ve 2 067 řádcích chybí cílová 
koncentrace a proto byly tyto řádky ihned odstraněny. Všechny záznamy jsou snímany v čase, tedy známe datum a čas měření, z něhož
jsme vytvořili další atribut den v týdnu. Pro regresi máme tedy k dispozici 12 atributů, kde vetšina je numerická. 
Tyto numerické atributy, spolu s cílovým atributem jsou shrnuty v Tabulce \ref{tab:numericColDesc}.

\begin{table}[ht!]
    \centering
    \begin{tabular}{l | r | r | r | r}

        Název atributu                                      & Průměr    & $\sigma$  & Min       & Max       \\\hline\hline
        Rosný bod (\degree C)                                      & 1,7502    & 14,4337   & -40,0000  & 28,0000   \\   
        Teplota (\degree C)                                        & 12,4016   & 12,1752   & -19,0000  & 42,0000   \\
        Tlak (hPa)                                          & 1016,4429 & 10,3007   & 991,0000  & 1046,0000 \\
        Rychlost větru (m/s)                                & 23,8667   & 49,6175   & 0,4500    & 565,4900  \\       
        Doba sněžení (hod.)                                 & 0,0553    & 0,7789    & 0,0000    & 27,0000   \\           
        Doba deště (hod.)                                   & 0,1949    & 1,4182    & 0,0000    & 36,0000   \\           
        Koncentrace $\text{PM}_{2.5}$ (\SI{}{\micro\gram}/$\text{m}^3$) & 98,6132   & 92.0504   & 0,0000    & 994,0000  \\           
    \end{tabular}
    \caption{Souhrn numerických atributů}
    \label{tab:numericColDesc}
\end{table}

Směr větru je kategoriální atribut, nabývající 4 různých hodnot, jejich distribuci můžeme vidět v grafu na Obrázku \ref{fig:wind_dir_dist}.
Originální měřená data obsahovala celkem 16 různých směrů větru, ale autoři je seskupili do 5, kde 4 nalezeneme v našem datasetu.
Hodnota CV znamená klidný, proměnný směr. Na Obrázku \ref{fig:target_by_wind_dir} můžeme vidět závislost cílové proměnné právě na směru větru,
všimneme si, že největší koncentrace pevných látek ve vzduchu je právě při klidném větru a naopak nejmenší, když vítr fouká severozápadně.
\image{wind_dir_dist.pdf}{Absolutní četnosti směru větru}{fig:wind_dir_dist}{0.7}
\image{target_by_wind_dir.pdf}{Koncentrace \pmQuant vzhledem ke směru větru}{fig:target_by_wind_dir}{0.9}

\subsection{Časové atributy}
Každý záznam v datasetu obsahuje informaci o dnu, měsící, roku a hodině, kdy byl vytvořen. My jsme dále přidali den v týdnu.
Všechny tyto atributy mohou být reprezentovány jako numerické nebo jako kategoriální. Aby jsme se mohli rozhodnout jak budeme
s těmity atributy pracovat podíváme se na krabicové grafy cílového atributu v závislosti na dnu, měsící atd.
Nejprve se podíváme na závislost pro dny v týdnu, tu můžeme vidět na Obrázku \ref{fig:target_by_wd}.

\image{target_by_week_day.pdf}{Koncentrace \pmQuant vzhledem ke dnu v týdnu}{fig:target_by_wd}{1.0}

Zde nepozorujeme žádnou velkou závislost. Podobný výsledek můžeme vidět u dnu měsíce \ref{fig:target_by_day}, 
měsíců \ref{fig:target_by_month} i roků \ref{fig:target_by_year}.
Koncentrace cílového atributu kolísá a nemůžeme najít žádnou závislost, že by se vzrůstajícím dnem, měsícem koncentrace
\pmQuant rostla či klesala. Rozhodli jsme se tedy, že s atributy budeme pracovat jako s kategoriálními.
Navíc vytvoříme další dataset, kde den v měsící zcela odstraníme.

\image{target_by_month_day.pdf}{Koncentrace \pmQuant vzhledem ke dnu v měsící}{fig:target_by_day}{1.0}
\image{target_by_month.pdf}{Koncentrace \pmQuant vzhledem k měsící }{fig:target_by_month}{1.0}
\image{target_by_year.pdf}{Koncentrace \pmQuant vzhledem k roku}{fig:target_by_year}{1.0}

V prílohách ještě najdeme koncentrace vzhledem ke dnu v týdnu \ref{fig:target_by_wd_yearly} a měsící \ref{fig:target_by_month_yearly} 
zvlášť pro každý rok.

\FloatBarrier
\subsection{Distribuce hodnot a odlehlá pozorování numerických atributů}
Většina hodnot rosného bodu se pohybuje v rozmezí od -20 \degree C do 20 \degree C. V rosném bodě nepozorujeme
žádné odlehlé pozorování. Krabicový graf a histrogram můžeme vidět na Obrázku \ref{fig:dewpt_analys}. 
Dále můžeme vidět grafy pro teplotu a to na Obrázku \ref{fig:temp_analys}. Pro teplotu jsme také nenalezli
odlehlé pozorování, to stejné platí pro atmosferický tlak.

\image{dewpt.pdf}{Krabicový graf, histrogram rosného bodu}{fig:dewpt_analys}{0.8}
\image{temp.pdf}{Krabicový graf, histrogram teploty}{fig:temp_analys}{0.8}

Nejvíce odlehlých pozorování nalezneme pro rychlost větru, jedná se o 4 893 záznamů z celkových 41 757.
Tyto pozorování můžeme vidět na grafu v Obrázku \ref{fig:wndspeed_analys}. Na histogramu vedle si všimneme,
že rychlost větru je většinou do 100 m/s. Odlehlé pozorování tedy odpovídají extrémum, kdy se do Pekingu dostal silný vítr.
Tak stejně odlehlé pozorování najdeme u doby sněžení a deště. Obrázky \ref{fig:hsnow_analys} a \ref{fig:hrain_analys}.
Drtivou většinu času v Pekingu nesněží ani neprší více jak hodinu v kuse, proto jsou hodnoty od 1 - 2 hodin označeny za odlehlé 
pozorování. Dlé mého názoru by bylo špatné rozhodnutí tyto hodnoty odstranit, neboť právě ony mohou značit
určitý pokles či vzrůst koncentrace \pmQuant

\image{wndspd.pdf}{Krabicový graf, histrogram rychlost větru}{fig:wndspeed_analys}{0.8}
\image{hsnow.pdf}{Krabicový graf, histrogram doba sněžení}{fig:hsnow_analys}{0.8}
\image{hrain.pdf}{Krabicový graf, histrogram doba deště}{fig:hrain_analys}{0.8}

Pro cílový atribut koncentrace \pmQuant bylo nalezeneo 1773 odlehlých pozorování, jak můžeme vidět v Obrázku \ref{fig:pm25_analys}.
Tyto odlehlé pozorování budou odstraněny pro všechny datasety, pro které budeme zkoušet regresi.
Odlehlé pozorování z předchozích atributů odstraníme v nově vytvořeném datasetu a vyzkoušíme jaký vliv
bude mít jejich přitomnost či absence. 

\image{pm25.pdf}{Krabicový graf, histrogram koncentrace \pmQuant}{fig:pm25_analys}{0.8}

\section{Předzpracování datasetu}
Zde si popíšeme operace, které jsme provedli s originálním datasetem, aby jsme získali naše datasety, nad kterými budeme
zkoušet regresi. 
První dataset jsme získali tak, že jsme odstranili odlehlé pozorování cílového atributu \pmQuant a provedli binarizaci atributů
den v týdnu, den v měsící, měsíc, rok a hodina. Výsledný dataset jsme následně normalizovali do rozmezí 0,0 až 1,0.
Pro druhý dataset jsme provedli to stejné ale odstranili jsme den měření, následně jsme provedli normalizaci.
Pro třetí dataset jsme odstranili všechny odlehlé pozorování, provedli binarizaci atributů a následnou normalizaci.
Z tohoto datasetu jsme následně pomocí redukce dimenze vytvořili ještě další 2 datasety, kterým jsme nechali 20 resp. 10
nejdůležitějších atributů. Souhrn datasetu nalezeneme v Tabulce \ref{tab:regression_datasets}.

\begin{table}
    \centering
    \begin{tabular}{l | r | r}
        Jméno                       & Počet transakcí   & Počet attributů \\\hline\hline
        df\_binAll                   & 39984             & 89 \\
        df\_binNoDay                 & 39984             & 58 \\
        df\_noOut\_binNoDay           & 33512             & 58 \\
        df\_noOut\_binNoDay\_20attr    & 33512             & 20 \\  
        df\_noOut\_binNoDay\_10attr    & 33512             & 10 \\
    \end{tabular}
    \caption{Souhrn datasetů pro regresi}
    \label{tab:regression_datasets}
\end{table}

\section{Regrese}

\bibliography{citations}
\bibliographystyle{ieeetr}

\FloatBarrier
\newpage
\section{Přílohy}

\image{target_by_wd_by_year.pdf}{Koncentrace \pmQuant vzhledem ke dnu v týdnu pro každý rok}{fig:target_by_wd_yearly}{1.0}
\image{target_by_month_by_year.pdf}{Koncentrace \pmQuant vzhledem k měsíci pro každý rok}{fig:target_by_month_yearly}{1.0}

\end{document}