\documentclass[a4paper,12pt]{article}

\usepackage[section]{placeins}
\usepackage[czech]{babel}
\usepackage[utf8]{inputenc}
\usepackage[T1]{fontenc}
\usepackage{listings}
\usepackage{hyperref}
\usepackage{enumitem}
\usepackage{mathrsfs}
\usepackage{enumitem}
\usepackage{amsmath}
\usepackage{amssymb}
\usepackage{graphicx}

\newcommand{\image}[4]{\begin{figure}[h!] \centering \includegraphics[width=#1\linewidth]{#2} \caption{#4} \label{#3} \end{figure}}

\author{Moravec Vojtěch}
\title{Analýze sítě \emph{Spoluautorství}}
\date{Letní semestr 2019}

\begin{document}
\maketitle
\newpage

\section{Popis datasetu}

Analyzovaný dataset byl získán ze stránky \emph{snap.stanford.edu} \cite{snapnets}. 
Dataset obsahuje seznam hran neorientovaného grafu, který prezentuje spolupráci vědců, kteří se zabývají obecnou relativitou a kvantovou kosmologií.
Hrana $(u,v)$ značí, že vědec $u$ se podílel na článku s vědcem $v$. Tento dataset obsahuje některé hrany vícekrát, ale pro naší analýzu nebereme v potaz
vážené hrany, kdežto smyčky, které se zde nacházení ponecháváme v sítí. Celkem je v této citační síti 12 smyček.

Data pro vytvoření datasetu byla nasbírána v průběhu let od Ledna roku 1993 do Dubna 2014, jedná se tedy o celkem 124 měsíců, hrany avšak neobsahují
časová razítka jejich vzniku.
 
\section{Analýza sítě}
\subsection{Analýza distribuce stupňů vrcholů}
Citační síť se skládá z $5242$ vrcholů a $14496$ hran, tedy $5242$ autorů a $14496$ zaznamenaných spoluautorství. V tabulce \ref{tab:vertexDeg} vidíme
shrnutí číselných charakteristik stupňě vrcholu. Maximální stupeň je roven $81$, ten říká, že nejaktivnější autor je spoluautorem 81 článků,
kdežto $1196$ autorů se podílelo pouze na jednom článku, v rámci obecné relativity a kvantové kosmologie. V průměru se autoři podílejí na 5-ti až 6-ti článcích.
Pokud by jsme ze sítě vyloučili smyčky, minimální stupeň by se změnil na $0$.

\begin{table}[h!]
    \centering
    \begin{tabular}{r | r | r | r | r}
        Minimum & Četnost       & Maximum   & Četnost       & Průměr \\\hline\hline
        1       & $1196 \times$ & 81        & $1 \times$    & 5,530  
    \end{tabular}
    \caption{Souhrn stupňů vrcholů}
    \label{tab:vertexDeg}
\end{table}

V grafu na Obrázků \ref{fig:relDegDist} vidíme relativní distribuci stupňů a na Obrázku \ref{fig:cummDegDist} kumulativní distribuci stupňů.
Podle relativní distirbuce bychom chtěli rozhodnout, zda distribuce odpovídá mocninnému rozdělení. Konec této relativní distribuce vykazuje mocninné dělení,
ikdyž celkově tato distribuce není tak čistá jak bychom si přáli.

\image{0.7}{../grafy/relDD.pdf}{fig:relDegDist}{Relativní distribuce stupňů}
\image{0.7}{../grafy/cumDD.pdf}{fig:cummDegDist}{Kumulativní distribuce stupňů}

\FloatBarrier
\newpage

\subsection{Analýza cest v síti}
Průměr sítě neboli, nejdelší z nejkratších cest mezi vrcholy, je v naší sítí roven 17. Průměrná délka nejkratší cesty je rovna $6,049$. Autoři článku jsou 
tedy průměrně propojeni přes cca $6$ další autorů. V grafu na Obrázku \ref{fig:eccentricity} můžeme vidět excentricitu jednotlivých vrcholů.
Excentricita je vzdálenost nejkratší cesty k nejvzdálenějšímu vrcholu.

\image{0.7}{../grafy/eccentricity.pdf}{fig:eccentricity}{Excentricita vrcholů}

Cesta v naší síti značí posloupnost autorů, kteří spolupracovani na určitém článku. Nulovou excentricitu má pouze jeden vrchol, tento vrchol můsí mít pouze
smyčku sám na sebe, neboť minimální stupeň je $1$ viz Tabulka \ref{tab:vertexDeg}, nikdy nespolupracoval s žádným dalším autorem.

\FloatBarrier
\newpage
\subsection{Centrality v síti}
Na Obrázcích \ref{fig:closeness}, resp. \ref{fig:betweenness} můžeme vidět closeness, resp. betweenness centralitu jednotlivých vrcholů.
Closeness centralita měrí, jak je vrchol centrální v dané síťi, bere v potaz nejkratší cesty ke všem ostatním vrcholům.
Dle closeness centrality jsou vrcholy rozděleny do dvou skupin, ačkoli ani jedna nedosohuje vysoké hodnoty této centrality. Tyto velmi nízky hodnoty jsou
způsobeny tím, že daný graf není souvislý a obsahuje 355 komponent, kde největší obsahuje 4158 vrcholů, nejmenší 1 vrchol a průměrné je v jedné komponentě
$14,767$ vrcholů.
Betweenness centralita měrí poměr nejkratších cest, všech dvojic vrcholů, které procházeji přes zkoumaný vrchol. Vrcholem s vysokou betweenness centralitou
prochází mnoho nejkratších cest a pokud jej odstranímě je největší šance, že dojde k rozpadu komponenty.

\image{0.7}{../grafy/closeness.pdf}{fig:closeness}{Closeness centralita vrcholů}
\image{0.7}{../grafy/betweenness.pdf}{fig:betweenness}{Betweenness centralita vrcholů}

\FloatBarrier
\newpage
\subsection{Shlukování}
V naší sítí je globální shlukovací roven $0,630$. Shlukovací koeficient jednotlivých vrcholů je už 
velmi různorodý jak můžeme vidět v grafu na Obrázku \ref{fig:cluster}. Vysoká tranzitiva vrcholů znamená, že jeho sousedé jsou s vysokou pravděpodobností taky spojeni
hranou.

\image{0.7}{../grafy/shlukovaci_koef.pdf}{fig:cluster}{Shlukovací koeficient vrcholů}
\FloatBarrier
\newpage
\subsection{Odolnost síťě}
V grafu na Obrázku \ref{fig:failAttack} můžeme vidět jak se měnili různé charakteristiky síťě při jejim selhání resp. útoku.
Při selháni je náhodně odebrán vrchol, kdežto u útoku je cíleně odebrán nejdůležitější vrchol, tedy ten s nejvyšším stupňěm. Grafy popisují změnu
charakteristik pro $1000$ iterací, bylo tedy odebráno celkem $1000$ vrcholů.

\image{1}{../grafy/failureAttack1000.pdf}{fig:failAttack}{Odolnost síťě vůči selhání a útoku}

Všimneme si, že při útoku rychle roste počet komponent, to způsobuje vznik menších komponent, což se odráží na průměrné vzdálenosti, která začne kolem 700-té iterace
klesat. Selhání většinou kopíruje vlastnosti útoku, ale mnohem pomaleji. Průměrná vzdálenost při selhání je ale vyjimkou, neboť ta se pořád drží velmi 
blízko originální hodnoty.
\FloatBarrier
\newpage
\subsection{Vzorkování}
Jak jsme zmínili naší síť tvoří $5242$ vrcholů, rozhodli jsme se vytvořit dva vzorky, které budou 25\% originální velikosti a zjístit zda jsou tyto vzorky
reprezentativní, tedy jejich charakteristiky jsou velmi podobé celé síti. První vzorek byl vytvořen metodou Random Node Sampling (RNS) a druhý Random Edge Sampling (RES).

\begin{table}[h!]
    \centering
    \begin{tabular}{l | r | r | r | r | r | r}
                                & Počet     & Počet & Prům.     & Prům.         & Shlukovací    & Prům. \\
                                & vrcholů   & hran  & stupeň    & vzdálenost    & koef.         & excentricita \\\hline\hline
        Originál            & 5242          & 14496 & $5,530$   & $6,049$       & $0,630$       & $9,557$       \\
        RNS                 & 1313          & 6197  & $9,440$   & $5,608$       & $0,813$       & $8,750$       \\
        RES                 & 1312          & 2371  & $3,614$   & $6,646$       & $0,679$       & $7,813$       \\
    \end{tabular}
    \caption{Charakteristiky vzorků}
    \label{tab:sampleChars}
\end{table}

Souhrn charakteristik originální síťě i dvou vzorků vidíme v Tabulce \ref{tab:sampleChars}, relativní distribuce vzorků pak na Obrázku \ref{fig:sampleDD}.
Obě metody vytvořili vzorky téměř stejné velikosti, ale metodou RES jsme přišli o cca $3 \times$ více hran, což se projevuje na průměrném stupni vrcholu. 
Průměrné vzdálenosti obou vzorku jsou velmi podobné originálu a taktéž shlukovací koeficient a méně pak i průměrná excentricita. Bohužel podle relativní
distribuce stupňů němůžeme jistě řící, že by vzorky měli distribuci stupňů podle mocninného zákona.

\image{0.7}{../grafy/sampleDD.pdf}{fig:sampleDD}{Relativní distribuce stupňů ve vzorcích}
\FloatBarrier
\newpage

\subsection{Hledání komunit}
Jak jsme již zmínili naše síť není souvislá a skládá se z 355 komponent, je tedy nemožné aby jsme našli méně než 355 komunit.
Pro hledání komunit jsme využili 4 různých metod, které jsou dostupné v R balíčku \emph{igraph}. Kvalita nalezených komunit je hodnocena podle jejich 
modularity.

\begin{itemize}
    \item Girvan-Newmanova metoda - divizivní metoda, kde při každé iteraci odebíráme hranu s největší betweenness centralitou, je velká šance, že tato hrana spojuje silně propojené komponenty, potencionální komunity.
    \item Fast-Greedy metoda - optimalizační metoda, maximalizující modularitu výběrem hustých podgrafů.
    \item Louvainova metoda - optimalizační metoda, hledání nejlepších komunit pomocí přesouvání vrcholů z jedné komunity do druhé. Vrchol bude vložen do komunity tak ať je maximalizována modularita.
    \item Walktrap metoda - hledá komunity pomocí náhodných procházek, v našem případě délky 5. Myšlenkou je, že krátké procházky zůstanou v komunitě daného vrcholu.
\end{itemize}

\begin{table}[h!]
    \centering
    \begin{tabular}{l | r | r}
        Algoritmus      & Počet komponent   & Modularita    \\\hline\hline
        Girvan-Newman   & 433               & $0,849$       \\
        Fast-Greedy     & 414               & $0,820$       \\
        Louvain         & 395               & $0,861$       \\
        Walktrap        & 698               & $0,790$       \\
    \end{tabular}
    \caption{}
    \label{tab:community}
\end{table}

Podle Tabulky \ref{tab:community} hodnotíme jako nejlepší Louvainův algoritmus, který dosáhl nejvěší modularity s nejmenším počtem komunit. Celkově všechny
algoritmy dosahují podobných modularit. Nejhůř vychází algoritmus Walktrap s velkým počtem komunit. Velikosti jednotlivých komunit dle algoritmů můžeme vidět
v Obrázcích \ref{fig:gnc}, \ref{fig:fgc}, \ref{fig:lc}, \ref{fig:wtc}.

\image{1}{../grafy/girvanNewmanCommunitySize.pdf}{fig:gnc}{Komunity podle Girvan-Newmanova algoritmu}
\image{1}{../grafy/fastGreedyCommunitySize.pdf}{fig:fgc}{Komunity podle Fast-Greedy algoritmu}
\image{1}{../grafy/louvainCommunitySize.pdf}{fig:lc}{Komunity podle Louvainova algoritmu}
\image{1}{../grafy/walkTrapCommunitySize.pdf}{fig:wtc}{Komunity podle Walktrap algoritmu}

\subsection{Šíření jevů}
Šíření jevů, např. nákazy v sítí se dá simulovat SIR modelem, kde v každém momentu je vrchol v jednom ze stavu:
\begin{itemize}
    \item S - nenakaženy, může být nakažen
    \item I - nakažený, může nakazit sousedy
    \item R - uzdraven, už nemůže být znovu nakažen
\end{itemize}

Na začátku se vybere náhodně určitý počet nakažených vrcholů, ty s určitou pravděpodobností nakazí své sousedy. Nakažený vrchol je uzdraven po určeném počtu kroků.
Uzdravený vrchol již nemůže být nakažen. Pokud bude na počátku vybrán vrchol s vysokým stupňěm tak se bude epidemie šířit rychle, pokud naopak s nizkým stupněm tak pomaleji.
Navíc, tím že síť je rozdělena do komponent mohou být určité komponenty zcela imunní pokud nebude nakažen žadný vrchol, který do nich patří.

\bibliography{citations}
\bibliographystyle{ieeetr}

\end{document}